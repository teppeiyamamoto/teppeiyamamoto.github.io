\documentclass[11pt]{article}

% == margins
\addtolength{\hoffset}{-0.75in} \addtolength{\voffset}{-0.75in}
\addtolength{\textwidth}{1.5in} \addtolength{\textheight}{1.6in}

% == this allows separate bibliography
\usepackage{bibunits}
\usepackage{natbib}

\renewcommand{\refname}{\vspace{-0.5in}} %adjust this number to get spacing right

%% === hyperref options ===
\usepackage{color}
\usepackage[bookmarks=true, bookmarksopen=true, linkcolor=webred]{hyperref}

%% === document starts here

\title{\bf POL 572: Quantitative Analysis II}
\author{\Large {\bf Spring 2009}}
\date{Kosuke Imai}

\begin{document}

\maketitle

\section{Contact Information}

\begin{flushleft}
\begin{tabular}{lcl}
Office: Corwin Hall 041 \\
Office Phone: 258--6601 \\
Email: \href{mailto:kimai@Princeton.Edu}{kimai@Princeton.Edu} \\
URL: \href{http://imai.princeton.edu}{http://imai.princeton.edu}\\
\end{tabular}
\end{flushleft}

\section{Logistics}

\begin{itemize}
\item Lectures: Wednesdays 9:00--10:20am in McCosh 62, and Thursdays
  10:30am--11:50am in Frist 307

\item Precepts (taught by Teppei Yamamoto
  \href{mailto:tyamamot@Princeton.EDU}{tyamamot@Princeton.EDU}):
  Monday 6pm -- 7:20pm in Friend 004

\item Kosuke's Office hours: Fridays 10am -- noon or stop by anytime 

\item Teppei's Office hours: Mondays \& Tuesdays 2pm -- 4pm in 235A Corwin
\end{itemize}

\section{Questions about the Course Materials}

In addition to precepts and office hours, please use the {\it
  discussion board} at Blackboard when asking questions about
lectures, problem sets, and other course materials.  This allows all
students to benefit from the discussion and help each other understand
the materials.  Teppei will be primarily responsible for handling
questions about precepts and problem sets, while I will answer the
questions about the lectures.  But, students are also encouraged to
participate in discussions and answer any quesions that are posted.

\section{Course Description}

\paragraph{Catalogue Description.} This is the second course in the
quantitative methods sequence. It will emphasize the flexibility of
the maximum likelihood framework in the context of regression models,
models that mix qualitative and continuous endogenous variables,
hazard models, and scaling models.

\paragraph{What this course is about.} This course is the first course
in applied statistical methods for social scientists. Students will
learn a variety of basic {\it cross-section} regression models (as
time permits!) including linear regression model, discrete choice
models, duration (or hazard) models, event count models, structural
equation models, and others.  Unlike traditional courses on applied
regression modeling, I will emphasize the connections between these
methods and causal inference, which is the primary goal of social
science research.


\section{Prerequisites}

There are three prerequisites for this course.
\begin{enumerate}
\item Mathematics covered in POL 502
\item Probability and statistics covered in POL 571
\item Statistical computing and programming covered in the statistical
  software workshop held at the end of January
\end{enumerate}

\section{Course Requirements}

The final grades are based on the following two items:
\begin{enumerate}
\item Problem sets (40\%): Although you are allowed to discuss the
  problem sets with others, you should not copy someone else's
  computer code or answers.  In particular, sharing a paper or
  electronic copy of your code and answers with other students is
  strictly prohibited.
 
\item Midterm exam (20\%): The midterm exam will be held around the
  spring break.

\item Final exam (40\%): It is also possible to conduct a
  final project in stead of taking the final exam if I view your
  proposed project appropriate for this course.  If you prefer this
  option, you must obtain my permission by the end of February.

\end{enumerate} 

\section{Computing}

In this course, we use a statistical computing environment, called R.
R is available for any platform and without charge at \href{
  http://www.r-project.org/}{http://www.r-project.org/}.  In a recent
{\it New York Times} article, R is described as software that ``allows
statisticians to do very intricate and complicated analyses without
knowing the blood and guts of computing systems.''

\section{Books}

\begin{enumerate}
\item Much of the following book will be assigned as required reading
  prior to the lectures:
   \begin{bibunit}[unsrtnat]
     \nocite{free:05}
      \putbib[my]
    \end{bibunit}


  \item In addition, you may find the following monographs useful:
    \begin{enumerate}
    \item Political Methodology
   \begin{bibunit}[unsrtnat]
     \nocite{king:98}
      \putbib[my]
    \end{bibunit}

    \item Probability and Statistics
   \begin{bibunit}[unsrtnat]
     \nocite{wass:05}
      \putbib[my]
    \end{bibunit}

    \item Econometrics
   \begin{bibunit}[unsrtnat]
     \nocite{wool:02}
      \putbib[my]
    \end{bibunit}

    \item R
   \begin{bibunit}[unsrtnat]
     \nocite{fox:02}
      \putbib[my]
    \end{bibunit}
    \end{enumerate}
\end{enumerate}

\section{Course Outline}

Each topic is followed by the list of required readings.  I assume
that students read these materials {\it prior to} each lecture.

\begin{enumerate}
\item Review, Introduction, and Overview
  \begin{enumerate}
  \item Causal inference, experimentation, and statistics 

      \medskip
      Chapter~1 of Freedman
      \medskip

  \item The potential outcomes framework of causal inference

  \begin{bibunit}[unsrtnat]
    \nocite{holl:86}
    \nocite{zhan:rubi:03}
    \putbib[my] 
  \end{bibunit}
  \medskip
  {\it *For Zhan and Rubin (2003), skip Sections 4 -- 7.} \\

  \item Randomized inference for classical randomized experiments
    
  \begin{bibunit}[unsrtnat]
    \medskip Chapter~2 of Paul Rosenbaum {\it Observational Studies}, 2nd ed.
    2002.   \nocite{ho:imai:06}
    \putbib[imai]
  \end{bibunit}
  \medskip    
  {\it *For Rosenbaum (2002), skip Sections 2.4.4, 2.5.4., 2.8, 2.9, and 2.10.} \\

  \item Classical randomized experiments and simple regression

    \begin{bibunit}[unsrtnat]
      \medskip
      Freedman, Chapter~2
      \nocite{free:08}
      \putbib[my]
    \end{bibunit}
    \medskip
    {\it *For Freedman (2008), skip Sections 5 -- 8.} \\

  \item Observational studies, parametric adjustment, and matching

    \begin{bibunit}[unsrtnat]
      \medskip
      \nocite{ho:imai:king:stua:07}
      \putbib[imai]
    \end{bibunit}

  \end{enumerate}


\item Multiple regression

  \begin{enumerate}
  \item Least squares

    \begin{bibunit}[unsrtnat]
      \medskip Chapter~4 of Freedman 
      \putbib[my]
    \end{bibunit}    
    \medskip {\it *If you are not comfortable with matrix algebra, you
      may refer to Chapter~3 of Freedman. } \\
    
  \item Maximum likelihood 

      \medskip Section~6.1 of Freedman 
      \medskip
    
  \end{enumerate}

\item Structural equation modeling, and instrumental variables

  \begin{enumerate}
  \item Selection problem, instrumental variables, and causality

    \begin{bibunit}[unsrtnat]
      \medskip Chapter~8 of Freedman
      \nocite{angr:imbe:rubi:96} 
      \putbib[my]
    \end{bibunit}    
 
  \item Direct and indirect effects

    \begin{bibunit}[unsrtnat]
      \medskip Chapter~5 of Freedman
      \nocite{imai:keel:yama:10} 
      \putbib[imai]
    \end{bibunit}    

  \end{enumerate}

\item Parametric regression modeling with various data types
  \begin{enumerate}
  \item Likelihood theory

    \medskip Chapter~4 of Gary King {\it Unifying Political
      Methodology} University of Michigan Press, 1998.

  \item Parametric and nonparametric Bootstrap

    \begin{bibunit}[unsrtnat]
      \medskip Chapter~7 of Freedman \nocite{king:tomz:witt:00}
      \putbib[my]
    \end{bibunit}    

  \item Binary data

    \medskip Chapter~6 of Freedman

  \item Other data types (as time permits)
    \begin{enumerate}
    \item Ordered and multinomial data
      
    \item Truncated and censored data
      
    \item Survival data
      
    \item Event count data
      
    \item Bivariate and multivariate data
      
    \item Data with nonrandom sample selection

    \end{enumerate}
  \end{enumerate}

\end{enumerate}

\end{document}
