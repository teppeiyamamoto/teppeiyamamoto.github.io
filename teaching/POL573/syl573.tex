\documentclass[11pt]{article}

% == margins
\addtolength{\hoffset}{-0.75in} \addtolength{\voffset}{-0.75in}
\addtolength{\textwidth}{1.5in} \addtolength{\textheight}{1.6in}


% == this allows separate bibliography
\usepackage{bibunits}
\usepackage{natbib}

\renewcommand{\refname}{\vspace{-0.5in}} %adjust this number to get spacing right

%% === hyperref options ===
\usepackage{color}
\usepackage[bookmarks=true, bookmarksopen=true, linkcolor=webred]{hyperref}

%% === document starts here

\title{\bf POL 573: Quantitative Analysis III}
\author{\Large {\bf Fall 2009}}
\date{Kosuke Imai}

\begin{document}

\maketitle

\begin{quote}
  This course is the second course in applied statistical methods for
  social scientists. Building on the materials we covered in POL 572
  or its equivalent (i.e., linear regression, structural equation
  modeling, instrumental variables, maximum likelihood estimation,
  discrete choice models), students will learn a variety of
  statistical methods including models for longitudinal data and
  survival data. Unlike traditional courses on applied regression
  modeling, I will emphasize the connections between these methods and
  causal inference, which is the primary goal of social science
  research.
\end{quote}

\section{Contact Information}

\begin{flushleft}
\begin{tabular}{lcl}
Office: Corwin Hall 036\\
Office Phone: 258--6601 \\
Email: \href{mailto:kimai@Princeton.Edu}{kimai@Princeton.Edu} \\
URL: \href{http://imai.princeton.edu}{http://imai.princeton.edu}\\
\end{tabular}
\end{flushleft}

\section{Logistics}

\begin{itemize}
\item Lectures: Mondays (101 Sherrerd Hall) and Thursdays (023
  Robertson Hall) 10:30--11:50am

\item Precepts (taught by Teppei Yamamoto
  \href{mailto:tyamamot@Princeton.EDU}{tyamamot@Princeton.EDU}): Tuesdays 4:30--6:00pm

\item Kosuke's Office hours: Stop by anytime or make an appointment by email

\item Teppei's Office hours: Wednesdays 4:30--6:30pm (126 Corwin)
\end{itemize}

\section{Questions about the Course Materials}

In addition to precepts and office hours, please use the {\it
  discussion board} at Blackboard when asking questions about
lectures, problem sets, and other course materials.  This allows all
students to benefit from the discussion and help each other understand
the materials.  Teppei will be primarily responsible for handling
questions about precepts and problem sets, while I will answer the
questions about the lectures and other course materials.  But,
students are also encouraged to participate in discussions and answer
any questions that are posted.

\section{Prerequisites}

There are three prerequisites for this course.
\begin{enumerate}
\item Mathematics at the level of the math camp and POL 502
\item Probability and statistics at the level of POL 571 and POL 572
\item Statistical computing and programming at the level of the
  statistical software workshop
\end{enumerate}

\section{Course Requirements}

Your final grade is based on the problem sets and the final project:
\begin{itemize}
\item {\bf Problem sets} (40\%): There are four problem sets.
  Although you are allowed to discuss the problem sets with others,
  you should not copy someone else's computer code or answers.  In
  particular, sharing a paper or electronic copy of your code and
  answers with other students is strictly prohibited.
 
%\item {\bf Inclass, closed-book exams} (60\%): Both the midterm and
%  final exams are inclass and closed-book.  The format of the exams
%  will be similar to those of POL 572.

\item {\bf Final project} (60\%): The final project is due 4pm,
  January 15.  Neither electronic nor late submission is allowed.  The
  detailed instructions are given below and will be discussed in the
  first lecture.
\end{itemize} 
% You may also choose to do both the exams and the final project, in
% which case your final grade will be based on the best of the two
% grades.

\section{Computing}

In this course, we support a statistical computing environment, called
R.  R is available for any platform and without charge at \href{
  http://www.r-project.org/}{http://www.r-project.org/}.  In a recent
{\it New York Times} article, R is described as 
\begin{quote}
  a popular programming language used by a growing number of data
  analysts inside corporations and academia.  It is becoming their
  lingua franca [...] whether being used to set ad prices, find new
  drugs more quickly or fine-tune financial models. Companies as
  diverse as Google, Pfizer, Merck, Bank of America, the
  InterContinental Hotels Group and Shell use it. [...] ``The great
  beauty of R is that you can modify it to do all sorts of things,''
  said Hal Varian, chief economist at Google. ``And you have a lot of
  prepackaged stuff that's already available, so you're standing on
  the shoulders of giants.''
\end{quote}
According to the article, R is also software that ``allows statisticians to do very intricate and complicated analyses without knowing the blood and guts of computing systems.'' 

We choose R for its flexibility and power.  However, students may use other statistical software such as STATA for the problem sets and the final project, but at their own risk; that is, we will not be able to answer your software-related questions.  Of course, there will be no penalty for using different statistical software.  What matters is the analysis you present rather than the software you use.

\section{Final Project}

The final project is the most important requirement of this course.
The goal is to conduct a project that can be {\it eventually}
developed into a high-quality published paper.  I encourage you to
continue working on your project after this course, and I am happy to continue to provide
guidance along the way.  You may also consider applying for the summer
political methodology meeting using the paper that comes out of your
final project.  In the past, papers based on final projects for this
course appeared in refereed journals and won a graduate student poster
award at the political methodology summer meeting.  Here are a few
important things to note when conducting the final project for this
course.
\begin{itemize}
\item {\bf Collaboration} with another student in this class is
  strongly encouraged for at least two reasons.  First, collaboration
  usually leads to a better project given the limited amount of time
  and intellectual capacity each student has; the sum is typically
  better than any of its parts and can be even better than the sum of
  its parts.  Second, collaboration will tend to make the project much
  more fun.  You will learn a great deal from each other and will also
  get to know your collaborator very well.  Please contact me if you
  are considering a collaborative project with someone who is not
  taking this class.

\item {\bf Simultaneous submission to another class} is allowed
  provided that you obtain a written permission from both instructors
  in advance.  This means that your project must be appropriate for both
  classes.

\item {\bf Two alternative approaches to the project}.  First, you
  may start with an original research question, develop a hypothesis,
  collect a data set, and conduct statistical analysis of the data
  in order to test the hypothesis.  That is, you conduct a research
  project from scratch.  The second approach is to start with the
  replication of the results in a published article and improve or
  extend the original analysis.  If you decide to take this approach,
  you need to contact the author(s) for the replication information as
  soon as possible (unless it is already available online) so that you
  will be able to get started on the analysis early in the semester.

\item {\bf Proposal} is due in class on December 3.  The proposal
  should contain {\it at the very minimum} the following components of
  the final project; (1) the brief statement of a question to be
  addressed and a hypothesis to be tested, (2) the detailed
  description of the data set to be analyzed, (3) the detailed results
  of a preliminary analysis, and (4) the detailed tentative plan of
  the final analysis.  For those of you who have made substantial
  progress, the ``proposal'' may take the form of a draft paper.

\item {\bf Writing style} is very important for the success of any
  scientific paper.  Be accurate and concise.  Pay special attention
  to tables and figures, which form the core of an empirical paper.
  Here is what I do when I draft a paper.  Start with tables, figures
  (and their detailed captions), and a rough description of the data
  and the methods you use (that is, do not start writing until you
  have them!).  Then, write the abstract followed by the introduction.
  Make sure that you clearly state your contributions in the abstract
  and at the beginning of the introduction.  Once these are done, then
  it is easy to write the rest of the paper by further elaborating
  each point you make in the introduction.  The structure of the paper
  should be top-down (i.e., don't keep readers guessing what you are
  going to say next), and you should carefully write the first and
  last sentences of each section and paragraph. Take advantage of sectioning to write a paper with a
  clear and logical structure. There are many 
  books and articles explaining how to write a good scientific paper,
  but here is one written by a political scientist.
  \begin{bibunit}[unsrtnat]
    \nocite{king:06}
    \putbib[my]
  \end{bibunit}

  Finally, I recommend that you consider using \LaTeX\ (together with BibTeX for bibliography).  


\item {\bf Evaluation} of the final project will be based on the
  following criteria.  For a purely methodological paper, the
  originality and significance of the proposed method will form the
  basis of evaluation.  The evaluation of a more applied paper depends
  on how a statistical method is effectively used to answer the
  substantive question set forth by its author(s).
\end{itemize}

\section{Textbooks}

There is no single textbook for this course.  Instead, the course
materials consist of lecture notes, lecture slides, and assigned
readings.  In addition, however, you may find the relevant parts of
the following textbooks useful.  Some of these books are reserved at
the Firestone library.
\begin{enumerate}
\item Political Methodology
  \begin{bibunit}[unsrtnat]
    \nocite{king:98}
    \putbib[my]
  \end{bibunit}
  
\item Probability and Statistics
  \begin{bibunit}[unsrtnat]
    \nocite{degr:sche:02,wass:05}
    \putbib[my]
  \end{bibunit}
  
\item Econometrics
  \begin{bibunit}[unsrtnat]
   \nocite{haya:00,wool:02}
   \putbib[my]
 \end{bibunit}
 
\item Regression Modeling
  \begin{bibunit}[unsrtnat]
   \nocite{gelm:hill:07}
   \putbib[my]
 \end{bibunit}

\item Causal Inference and Research Design
  \begin{bibunit}[unsrtnat]
    \nocite{morg:wins:07,angr:pisc:09}
    \putbib[my]
  \end{bibunit}
 
\item R
  \begin{bibunit}[unsrtnat]
    \nocite{fox:02}
    \putbib[my]
  \end{bibunit}
\end{enumerate}

\section{Tentative Course Outline}

We will cover the following topics in the order they are listed (and as time permits!).

\begin{enumerate}
\item Regression Modeling for Cross-section Data (Continued from POL 572)
  \begin{enumerate}
  \item Event count models 
  \item Generalized linear models
  \end{enumerate}

\item Survival Data Analysis
 \begin{enumerate}
 \item Basic concepts
 \item Parametric regression models
 \item Cox proportional-hazard model
 \item Competing risks models
 \end{enumerate}

\item Causal Inference with Cross-section Data
  \begin{enumerate}
  \item Matching methods
  \item Weighting methods
  \end{enumerate}

\item Regression Modeling for Longitudinal Data
  \begin{enumerate}
  \item Linear mixed effects models
  \item Generalized linear mixed effects models
  \item Multilevel, Hierarchical models
 \end{enumerate}

\item Causal Inference for Longitudinal Data
  \begin{enumerate}
  \item Difference-in-difference models
  \item Matching methods
  \item Weighting methods
  \end{enumerate}

\item Statistical Analysis with Missing Data
  \begin{enumerate}
  \item Basic concepts and assumptions
 \item Multiple imputation
  \end{enumerate}

\end{enumerate}

\end{document}
